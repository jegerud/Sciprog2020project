\section{Introduksjon}

\subsection{Forrord}
Prosjektgruppen består av tre studenter fra dataingeniørstudiet ved NTNU Gjøvik. I løpet av 
semesteret har vi introdusert for matematiske teknikker for numeriske løsninger av bestemte 
integral, ordinære og systemer av differensiallikninger og ikkelineære algebraiske likninger. 
Vi har også fått lære om høynivå programmering for lineæralgebra, optimalisering, bildebehandling 
og maskinlæring. Oppgaven som ble gitt innebærte at vi måtte benytte oss av de vi har lært i løpet
av semesteret. Selv om det har vært masse utfordringer underveis har vi fått masse praktisk 
læring. 

\subsection{Oppgavedefinisjon}

\subsection{Gruppens deltagere}
Gruppen vår består av tre studenter på dataingeniørstudiet ved NTNU Gjøvik. Siden de to første årene på
studiet utelukkende består av obligatoriske fag, har vi stort sett de samme kunnskapene og 
forutsetningene før prosjektets start. Uansett gjør det at vi er en gruppe på 3 at vi møter utfordringer
med ulike tilnærminger, og har vært viktig for å løse problemstillingene på en best mulig måte.

Vi har ikke benyttet Python som hovedprogrammeringsspråk i emner vi har deltatt i før dette semesteret,
men har før prosjektets start benyttet det i alle arbeidskrav i emnet. Dette gjør at vi har en viss 
kjennskap til hvordan Python fungerer og hvilke fordeler og ulemper dette medfører. Rapporten er skrevet 
med LaTeX, som er et språk vi har brukt på et mindre prosjekt tidligere på studiet. Dette har ikke
gitt oss noen utfordringer når vi har skrevet på rapporten

For oss har nok den største utfordringer med Python vært hvordan vi skal strukturere et såpass stort
prosjekt med flere forskjellige filtyper. 