\section{Kontrastforsterkning}
\label{sec:Kontrastforsterkning}
\subsection{Bakgrunn}
I et bilde er det kontrasten som gjør at vi kan skille et objekt fra bakgrunnen og eventuelle andre objekter.\cite{wiki:kontrast} Et godt eksempel på dette er svart tekst på hvit bakgrunn, hvor kontrasten mellom bokstavene og bakgrunnen er veldig stor. Dette gjør teksten lett å se og lese. I et bilde er det fargekontrasten som gjør at vi kan se objekter på et bilde. Et med bilde med høy motivkonstrast viser de lyse delene av bildet enda lysere og de mørke delene av bildet enda mørkere. Dette gjør at bildet ser mindre flatt ut for seeren, men gjør i realiteten at det blir mindre detaljer i bildet. Spesielt i de mørkeste og lyseste områdene blir bildene udetaljerte og støyete. Når man skal behandle et bilde vil det være veldig lett å øke kontrastene, mens muligheten for å redusere kontrasten vil være veldig begrenset. 

Kontrastforsterkning øker synligheten av et objekt ved øke forskjellen i lysmengde mellom objektet og bakgrunnen. En måte å gjøre dette på er å øke forskjellen i lysstyrke jevnt over i de dynamiske områdene, før man så øker forskjellen i mørke områder eller fremhever områder av bildet på bekostning av lysstyrkeforskjellene i andre områder. 

\subsection{Teori}

\subsection{Implementasjon}