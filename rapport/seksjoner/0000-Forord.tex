\section*{Forord}
Prosjektgruppen vår består av tre studenter fra dataingeniørstudiet ved NTNU Gjøvik. I løpet av 
semesteret har vi blitt introdusert for matematiske teknikker for numeriske løsninger av bestemte 
integral, ordinære og partielle differensiallikninger og systemer av disse, samt ikkelineære algebraiske likninger. Vi har også fått lære om høynivå programmering for lineæralgebra, optimalisering, bildebehandling og maskinlæring. Oppgaven som ble gitt hadde som hensikt at vi skulle benytte oss av alt vi har lært i løpet av dette semesteret. Tross mange utfordringer har vi fått et enormt læringsutbytte i form av teoretisk kunnskap og praktiske ferdigheter.

De to første årene av dataingeniørstudiet består utelukkende av obligatoriske fag. Derfor har gruppens medlemmer stort sett de samme forkunnskapene og forutsetningene før prosjektets start. En fordel med at vi er en gruppe på er 3 at vi møter utfordringer med ulike tilnærminger, og nettopp dette har vært essensielt for å løse problemstillingene som er blitt gitt på en best mulig måte.

Vitenskapelig Programmering er det første emnet vi deltar i hvor Python benyttes som programmeringsspråk, og vi på gruppa hadde derfor lite erfaring med Python før dette semesteret. Før prosjektets start har vi benyttet det i alle arbeidskrav som er blitt gitt i emnet. Dette har gitt oss en viss kjennskap til hvordan Python fungerer og hvilke fordeler og ulemper dette medfører. Vår største utfordring med Python har vært hvordan vi skal strukturere et såpass stort prosjekt med mange ulike filtyper. Ingen av gruppens medlemmer har tidligere erfaring med å utvikle stort annet enn applikasjoner som kjører i kommandolinje. Dette har ført til en bratt læringskurve og mye nyttig læring.

Rapporten er skrevet i \LaTeX, et språk vi har brukt på et mindre prosjekt tidligere på studiet. Selve skrivingen av rapporten har ikke gitt oss nevneverdige utfordringer. Derimot har det gitt oss utfordringer når rapporten måtte holdes lagret sammen med kildekoden i GitLab. Vi slet særlig når vi skulle finne en god editor for å jobbe på de ulike rapportfilene samtidig som det var en løsning som vi synes var effektiv. Vi endte opp med å skrive rapporten i Overleaf\footnote{\url{https://www.overleaf.com}} og pushe endringer til repositoriet med jevne mellomrom, slik at det skal være mulig å følge utviklingen i ettertid.

\subsection*{Git Repository}
Progresjonen på prosjektet ligger i Git-repositoriet på GitLab:

\url{https://git.gvk.idi.ntnu.no/casperfg/imt3881-2020-prosjekt}