\section{Forord}
Prosjektgruppen består av tre studenter fra dataingeniørstudiet ved NTNU Gjøvik. I løpet av 
semesteret har vi blitt introdusert for matematiske teknikker for numeriske løsninger av bestemte 
integral, ordinære og partielle differensiallikninger, og systemer av disse, samt ikkelineære algebraiske likninger. Vi har også fått lære om høynivå programmering for lineæralgebra, optimalisering, bildebehandling 
og maskinlæring. Oppgaven som ble gitt har innebåret at vi måtte benytte oss av de vi har lært i løpet
av semesteret. Tross mange utfordringer har vi fått et enormt læringsutbytte i form av teoretisk kunnskap og praktiske ferdigheter.

Gruppen vår består av tre studenter på dataingeniørstudiet ved NTNU Gjøvik. Siden de to første årene på studiet utelukkende består av obligatoriske fag, har vi stort sett de samme kunnskapene og forutsetningene før prosjektets start. Uansett gjør det at vi er en gruppe på 3 at vi møter utfordringer med ulike tilnærminger, og nettopp dette har vært essensielt for å løse problemstillingene på en best mulig måte.

Vi har ikke benyttet Python som hovedprogrammeringsspråk i emner vi har deltatt i før dette semesteret, men har før prosjektets start benyttet det i alle arbeidskrav i emnet. Dette gjør at vi har en viss kjennskap til hvordan Python fungerer og hvilke fordeler og ulemper dette medfører. For oss har nok den største utfordringer med Python vært hvordan vi skal strukturere et såpass stort prosjekt med flere forskjellige filtyper. Ingen av oss har vært borti utvikling av en GUI-applikasjon med Python og Qt som programmeringsspråk før. Dette har derfor gitt en bratt læringskurve og masse læring.

Rapporten er skrevet med LaTeX, som er et språk vi har brukt på et mindre prosjekt tidligere på studiet. Når vi har skrevet rapporten har det ikke gitt oss noen nevneverdige utfordringer. Det har derimot gitt oss utfordringer når rapporten måtte holdes lagret sammen med kildekoden i GitLab. Særlig når vi skulle finne en god editor for å jobbe på de ulike rapportfilene samtidig slet vi med å finne en løsning som vi synes var effektiv.

\subsection{Git Repository}
Progresjonen på prosjektet ligger i git-repositoriet på GitLab:

\url{https://git.gvk.idi.ntnu.no/casperfg/imt3881-2020-prosjekt}