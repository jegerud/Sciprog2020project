\newpage
\section{Anonymisering av bilder med ansikter}
\label{sec:Anonymisering}
\subsection{Bakgrunn}
Anonymisering av bilder er enkelte ganger nødvendig før det blir offentlig vist frem for å skjerme den avbildedes personvern. En teknikk som ofte blir brukt er å gjøre ansiktene uskarpe og resten av bildet forblir skarpt. Teknikken avhenger av at programmet klarer å gjenkjenne ansiktene\cite{wiki:FaceDetection} for å definere en maske rundt, slik at man kan isolere anvendingen av bildet kun på ansiktet, og i likhet som tidligere er det flere teknikker som benyttes. I denne oppgaven brukte vi biblioteket OpenCV (Open Source Computer Vision Library) og maskinlæringsalgoritmen Haar Cascade. 

\subsubsection{Open CV}
Open CV er et open source datasyn-\cite{datasyn} og maskinlæringsbibliotek som ble utviklet for å skape et felles infrastruktur for datasynsapplikasjoner og for å akselerere bruken av maskinoppfatning i kommersielle produkter\cite{cv2}. Kort fortalt betyr dette datamaskinens evne til å tolke data på en måte som ligner på hvirdan mennesker bruker sansene til å oppfatte verden rundt seg\cite{wiki:machinePerception}, og da spesielt synet i dette tilfellet.  

\subsubsection{Haar Cascade algoritmen}
Haar Cascade er en maskinlæringsalgoritme~\cite{haar} som brukes til å identifisere objekter i et bilde eller i en video. Den er mest kjent til å identifisere ansikter, men kan trenes til å identifisere nesten hva det måtte være av objekter.

Algoritmen trenger å trenes med flere positive bilder av ansikter og negative bilder uten ansikter, deretter må man trekke ut kjente egenskaper fra disse bildene. Første steget er å samle inn Haar-features. 

De to første elementene (figur \ref{fig:haarfeat}) kalles for edge-features som benyttes til å detektere kanter i objekter. Den tredje er en line-feature, mens den fjerde heter four rectangle-feature som oftest benyttes til å gjengkjenne en skrå linje.

Haar-Features er veldig god til å gjenkjenne linjer og kanter på objekter, og spesielt i ansikter. Som man ser i (figur \ref{fig:haarface}) klarer en edge-feature å gjengkjenne essensielle deler ved et ansikt, som f.eks et øye. Grunnen til dette er at i bildet blir det en kontrast mellom øyet og kinnet, som kan minne om objekt 2 fra figur (\ref{fig:haarfeat}), altså en mørk overdel med lys bunn. Algoritmen går så gjennom hele bildet helt til det oppdager en samling av slike "features" og ved nok positive treff markerer den det område over ansiktet.

Algoritmen trenger to paramtere som heter scale factor og minimum neighbor

\subsection{Implementasjon}
Det mest krevende var nettopp å få programmet til å gjenkjenne ansikter. Viktigste var å importere xml-filen\footnote{haarcascade\_frontalface\_default.xml\cite{xml:haar}} som inneholdt en ferdig trent algoritme for gjenkjenning av ansikter som er vendt direkte mot kamera. Valg av paramterene måtte tas basert på bildet.
\import{./kode/}{detectFace}

\begin{figure}
\begin{center}
    \includegraphics[width=0.4\columnwidth]{bilder/Anonymisering/VJ_featureTypes.svg.png}
     \caption{Haar Feature\label{fig:haarfeat}} \source{Wikimedia Commons~\cite{wiki:haarfeat}}
\end{center}
\end{figure}

\begin{figure}
\begin{center}
    \includegraphics[width=0.5\columnwidth]{bilder/Anonymisering/haarface-example.png}
     \caption{Haar Feature demonstrert på et ansikt \label{fig:haarface}} \source{Wikimedia Commons~\cite{wiki:haarface}}
\end{center}
\end{figure}
 
%\import{./kode/}{anonymiseringModul}
%\begin{center}
%\includegraphics[scale = 0.3]{./bilder/Anonyme/lena.png}
%\includegraphics[scale = 0.28]{./bilder/Anonyme/lenaBlur.png}
%\end{center}