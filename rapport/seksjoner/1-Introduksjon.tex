\section{Introduksjon}


\subsection{Rapportens oppbygging}
Vi har valgt å følge rapportmalen til NTNU ***. Vi la malen inn i den nettbaserte LaTeX-editoren Overleaf, som vi tilslutt endte opp med å bruke til å skrive rapporten. Når vi i rapporten skal vise til andre kapitler og seksjoner vil det bli brukt kapittelets og seksjonens navn og nummer. For listformer som f.eks. punktlister vil vi bruke terminologien punkt. Vedlegg blir lagt bakerst i rapporten og vil bli referert til med vedleggets bokstav og navn. Rapporten har følgende oppbygning:
\begin{itemize}
  \item[-] Introduksjon - beskriver raskt omstendighetene rundt prosjektet. Gir også en kort beskrivelse av gruppen og rapporten.
  \item[-] Oppgaven - beskriver oppgaven som er gitt, hva som er forventet og hvordan det kan løses.
  \item[-] Løste oppgaver - viser hvordan vi har løst oppgavene, hvordan vi har tenkt og hvilke forutsetninger som er gjort
  \item[-] Avslutning - refleksjoner rundt prosjektet blir gjort. Det blir også presentert diskusjoner og evaluering som gjort underveis.
  \item[-] Konklusjon - inneholder avsluttende tanker, hva som fungerte og hva som kunne vært gjort annerledes.
  \item[-] Bibliografi - inneholder alle kilder vi har benyttet i løpet av prosjektets periode.
  \item[-] Vedlegg - inneholder kode som vi ønsket å forklare nærmere.
\end{itemize}

\subsection{Oppgaven}
En rekke problemer i bildebehandling kan løses med en teknikk som kalles «Poisson Image Editing». Metoden går i korthet ut på at man representerer bildet man ønsker å komme frem til som en funksjon $u : \Omega \to C$, der $\Omega \subset \mathbb{R}^2$ er det rektangulære området hvor bildet er definert, og $C$ er fargerommet, vanligvis $C = [0, 1]$ for gråtonebilder og $C = [0, 1]^3$ for fargebilder. Bildet $u(x, y)$ fremkommer som en løsning av Poisson-ligningen
\begin{equation}
  \frac{\partial^2 u}{\partial x^2} + \frac{\partial^2 u}{\partial y^2} \equiv \nabla^2 u = h,
  \label{eq:poisson}
\end{equation}
der randverdier på $\partial\Omega$ og funksjonen $h : \Omega \to \mathbb{R}^{\dim(C)}$ spesifiseres avhengig av hvilket problem som skal løses. Randverdiene er av Dirichlet- eller Neumann-typen.

En måte å løse Poisson-ligningen på er å iterere seg frem til løsningen vha. såkalt gradientnedstigning («gradient descent»). I praksis gjøres dette ved å innføre en kunstig tidsparameter og la løsningen utvikle seg mot konvergens:
\begin{equation}
\frac{\partial u}{\partial t} = \nabla^2 u - h.
\label{eq:diffusjon}
\end{equation}
Når man velger denne fremgangsmåten, må man også velge en initialverdi for bildet, $u(x, y, 0) = u_0(x, y)$.

To diskrete numeriske skjemaer for~(\ref{eq:diffusjon}) kan finnes ved henholdsvis eksplisitt og implisitt tidsintegrasjon og sentrerte differanser for de partielle deriverte:
\begin{align}
  \frac{u^{n+1}_{i,j} - u^n_{i,j}}{\Delta t} &= \frac{1}{\Delta x^2}
                                               (u^n_{i+1,j} +
                                               u^n_{i-1,j} +
                                               u^n_{i,j+1} +
                                               u^n_{i,j-1} -4 
                                               u^n_{i,j}) - h_{i,j},
                                               \label{eq:eksplisitt}  \\
  \frac{u^{n+1}_{i,j} - u^n_{i,j}}{\Delta t} &= \frac{1}{\Delta x^2}
                                               (u^{n+1}_{i+1,j} +
                                               u^{n+1}_{i-1,j} +
                                               u^{n+1}_{i,j+1} +
                                               u^{n+1}_{i,j-1} -4 
                                               u^{n+1}_{i,j}) - h_{i,j}.
                                               \label{eq:implisitt}
\end{align}

Fra dette skulle vi implementere en rekke anvendelser på bilder som for hver anvendelse blir forklart nøyere i denne rapporten. Oppgaveteksten inneholdt noen eksempler og generelle forklaringer på hvordan man kan implementere anvendelsene, men vi fikk også frihet til å inkludere egne anvendelser.

Det ble definert en minimumsløsning som skulle inneholde implementasjon av det eksplisitte skjemaet~(\ref{eq:eksplisitt}) og anvendelsene glatting (avsnitt~\ref{sec:Glatting}), inpainting (avsnitt~\ref{sec:Inpainting}), kontrastforsterkning for gråtonebilder (avsnitt~\ref{sec:Kontrastforsterkning}) og sømløs kloning (avsnitt~\ref{sec:kloning})

