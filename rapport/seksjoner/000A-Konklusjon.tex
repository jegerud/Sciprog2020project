\newpage
\section{Konklusjon}
Alle gruppens medlemmer har opplevd prosjektet som en nyttig erfaring, både rent faglig og personlig. I tillegg synes vi at det omhandler et interessant tema, noe som har gjort at vi har lagt inn en ekstra innsats. Vi har fått kjenne på frustrasjon over å ikke få til enkelte elementer, men samtidig en tilsvarende glede når det endelig har løsnet. Det oppstod et uforutsett problem i og med at all arbeid på prosjektet måtte foregå adskilt fra hverandre over nettet på grunn av korona-pandemien som stengte de fleste landene i verden. Til tross for dette fungerte prosjektorganisasjonen bra på grunn av ukentlig møter der vi diskuterte arbeidet som hadde blitt gjort den foregående uka, og hva som skulle gjøres videre. Det hadde antakelig vært enklere å gjennomføre prosjektet om man fysisk kunne jobbe sammen på campus, men det oppstod ingen nevneverdige problemer til tross for at dette ikke var mulig.

%- Brancher og mergerequests
Ved prosjektstart var vår kjennskap til \texttt{Git}\footnote{\url{https://git-scm.com/}} stort sett begrenset til \texttt{Git pull}, \texttt{Git commit} og \texttt{Git push}. Med disse forutsetningene bestemte vi oss for å bruke én branch når vi startet med prosjektet, noe som fort viste seg å føre til mange mergeconflicter. Det ble vurdert å opprette indivuduelle brancher, men bestemte oss for å ikke gjøre det da prosjektet allerede var godt i gang. Derfor bestemte vi oss for å opprette en \texttt{develop}- branch i tillegg til \texttt{master}- branchen. Her var tanken at all implementasjon av ny kode og funksjonaliteter skulle gjøres i \texttt{develop}, mens \texttt{Master} hele tiden skulle bestå av kjørbar kode. Dette gjorde at vi enkelt kunne følge de andres progresjon og på den måten kunne hjelpe hverandre eller komme med egne tanker dersom noe ikke fungerte.

%- GUI
Siden ingen på gruppen hadde erfaring med PyQt5 har bruken av dette når vi har utviklet GUI medført en bratt læringskurve. Det har også ført til en del erfaringer som vi tar med oss videre og det bemerkes at vi ville gjort en del ting annerledes dersom vi skulle gjort dette på nytt. Først og fremst ville vi droppet å bruke QtDesigner til å lage GUI og heller skrevet designet med kode. Dette ville gitt oss mer kontroll over hvordan GUI-et ville oppført seg, og vi kunne f.eks. da enkelt implementert støtte for å kunne justere vinduer ved å dra i kantene av vinduet. Ved å droppe QtDesigner ville vi også unngått bruk av \texttt{.ui}-filer som førte til unødvendige restriksjoner på organiseringen av prosjektmappa. Vi visste ikke helt hvordan GUIet vil bli ved prosjektstart, og mange av løsningene er kommet på underveis. Dette har ført til en del kompliserte løsninger og at vi flere steder ender opp med å bygge på og bygge på. Spesielt lagring av bilder ble ganske rotete og unødvendig komplisert til slutt. Et annet problem med lagringsfunksjonen er at pikselverdier mellom 0 og 1 må konverteres til pikselverdier mellom 1 og 256 før det lagres, noe som fører til at noen bilder blir mer glattet enn de egentlig skulle bli.

Når det gjelder gjenbruk av kode, kunne dette blitt løst på en bedre måte. Spesielt har vi implementert en rekke eksplisitte skjemaer som er nesten identiske. Forbedringspotensialet her vil være å holde seg til ett skjema, og heller sende med \texttt{h}, som i de fleste tilfellene var den eneste forskjellen, som en parameter. Dermed kunne mange linjer blitt erstattet med et enkelt funksjonskall. 

Alt i alt er vi fornøyde med arbeidet som har blitt gjort på prosjektet. Det har vært til tider veldig krevende og frustrerende, men som raskt blir erstattet med en stor mestringsfølelse når alt fungerer som det skal. Dette prosjektet har gitt oss masse god kunnskap og erfaring rundt prosjektorganisering, bruk av Git-repository, \LaTeX, Python og selvfølgelig Poisson Image Editing.