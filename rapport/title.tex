\begin{titlepage}
\vbox{ }
\vbox{ }
\begin{center}
% Upper part of the page
\includegraphics[width=0.50\textwidth]{images/NTNU_logo.png}\\[0.7cm]
\textsc{\LARGE IMT3881 - Vitenskapelig Programmering}\\[0.4cm]
\vbox{ }
% Title
\HRule \\[0.4cm]
{\huge \bfseries Poisson Image Editing - Prosjektoppgave}\\[0.4cm]
\HRule \\[1.5cm]

\large
Anders Kjelsrud \\
Casper F. Gulbrandsen \\
Kristian Jegerud \\

\vspace{20}
\begin{abstract}
    Emnet IMT3881 Vitenskapelig Programmering kombinerer realfaglige teknikker og høynivå programmering, og gir studentene et innblikk i hvordan dette kan benyttes til å løse hverdagslige problemsstillinger. Et godt eksempel på dette er bildebehandling, hvor numeriske løsninger av differensiallikninger og ikkelineære algebraiske likninger sammen med biblioteker fra Python kan benyttes til å behandle bilder. Hovedoppgaven vår har vært å implementere ulike anvendelser for dette, både for fargebilder og gråtonebilder. Vi har også skrevet et lite program med grafisk brukergrensesnitt (GUI) som viser eksempler på anvendelsene vi har implemntert. Ved å lese denne rapporten får du innsikt i hvordan vi har jobbet, hvilke teknikker som er brukt og hvordan disse er benyttet. All kildekoden ligger tilgjengelig på GitLab\cite{gitsource}. 
\end{abstract}
\vfill
% Bottom of the page
{\large Våren, 2020}
\end{center}
\end{titlepage}