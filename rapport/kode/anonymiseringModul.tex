\begin{lstlisting}[language=Python]
def blurFace(file, scaleFactor = 1.2, minNeighbors = 5):
    """
    Oppdager et ansikt vendt mot kamera
   
    Parameters + some explanation
    ---------
    file         : Bildefil
                    Pathen til filen der originalbildet er
    scaleFactor  : int
                    Kompenserer i tilfelle noen ansikter er lenger unna enn andre
    minNeighbors : int
                    ----
    title        : text
                    Tittelen til bildet som er anvendt
    """
    image = cv2.imread(file) #leser inn bildet
    image = cv2.cvtColor(image, cv2.COLOR_BGR2RGB)

    #importerer haarscade biblioteket
    face_cascade = cv2.CascadeClassifier('haarcascade_frontalface_default.xml')  
    faces = face_cascade.detectMultiScale(image,
                                          scaleFactor,
                                          minNeighbors,
                                          minSize = (30,30))
    for (x,y,w,h) in faces:
        RoI = image[y:y+h, x:x+w]            #Region of Interest --> ansiktet
        RoI = RoI.astype(dtype = float)
        blur = eks.eksplisittAnonym(RoI,image[y:y+h, x:x+w], 0.25,250)               
        image[y:y+h, x:x+w] = blur
        
    return len(faces), image
    \end{lstlisting}

\begin{lstlisting}[language=Python]  
import AnonymiseringModul as anonym
import ImageView as iv

lena = '../hdr-bilder/faces/lena.png'
family = '../hdr-bilder/faces/family.jpg'
group = '../hdr-bilder/faces/group.jpg'

antall, image = anonym.blurFace(lena)
print(antall, "ansikt er registrert")   #teller opp antall ansikt funnet
iv.singleView(image, "Lena")            #displayer bildet
\end{lstlisting}


    