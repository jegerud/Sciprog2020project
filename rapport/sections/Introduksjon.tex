\section{Introduksjon}

\subsection{Forrord}
Prosjektgruppen består av tre studenter fra dataingeniørstudiet ved NTNU Gjøvik. I løpet av 
semesteret har vi introdusert for matematiske teknikker for numeriske løsninger av bestemte 
integral, ordinære og systemer av differensiallikninger og ikkelineære algebraiske likninger. 
Vi har også fått lære om høynivå programmering for lineæralgebra, optimalisering, bildebehandling 
og maskinlæring. Oppgaven som ble gitt innebærte at vi måtte benytte oss av de vi har lært i løpet
av semesteret. Selv om det har vært masse utfordringer underveis har vi fått masse praktisk 
læring. 

\subsection{Gruppens deltagere}
Gruppen vår består av tre studenter på dataingeniørstudiet ved NTNU Gjøvik. Siden de to første årene på studiet utelukkende består av obligatoriske fag, har vi stort sett de samme kunnskapene og forutsetningene før prosjektets start. Uansett gjør det at vi er en gruppe på 3 at vi møter utfordringer med ulike tilnærminger, og har vært viktig for å løse problemstillingene på en best mulig måte.

Vi har ikke benyttet Python som hovedprogrammeringsspråk i emner vi har deltatt i før dette semesteret, men har før prosjektets start benyttet det i alle arbeidskrav i emnet. Dette gjør at vi har en viss kjennskap til hvordan Python fungerer og hvilke fordeler og ulemper dette medfører. For oss har nok den største utfordringer med Python vært hvordan vi skal strukturere et såpass stort prosjekt med flere forskjellige filtyper. Ingen av oss har vært borti utvikling av en GUI-applikasjon med Python og Qt som programmeringsspråk før. Dette har derfor gitt en bratt læringskurve og masse læring.

Rapporten er skrevet med LaTeX, som er et språk vi har brukt på et mindre prosjekt tidligere på studiet. Når vi har skrevet rapporten har det ikke gitt oss noen nevneverdige utfordringer. Det har derimot gitt oss utfordringer når rapporten måtte holdes lagret sammen med kildekoden i GitLab. Særlig når vi skulle finne en god editor for å jobbe på de ulike rapportfilene samtidig slet vi med å finne en løsning som vi synes var effektiv.

\subsection{Rapport}
Vi har valgt å følge rapportmalen til NTNU ***. Vi la malen inn i den nettbaserte latexeditoren overleaf, som vi tilslutt endte opp med å bruke til å skrive rapporten. Når vi i rapporten skal vise til andre kapitler og seksjoner vil det bli brukt kapittelets og seksjonens navn og nummer. For listformer som f.eks. punktlister vil vi bruke terminologien punkt. Vedlegger blir lagt bakerst i rapporten og vil bli referrert til med vedleggets bokstav og navn. Rapporten har følgende oppbygning:
\begin{itemize}
  \item[1] Introduksjon - beskriver raskt omstendighetene rundt prosjektet. Gir også en kort beskrivelse av gruppen og rapporten.
  \item[2] Oppgaven - beskriver oppgaven som er gitt, hva som er forventet og hvordan det kan løses.
  \item[3 - 11] Løste oppgaver - viser hvordan vi har løst oppgavene, hvordan vi har tenkt og hvilke forutsetninger som er gjort
 ++  \item[12] Avslutning - reflektesjoner rundt prosjektet blir gjort. Det blir også presentert diskusjoner og evaluering som gjort underveis.
  \item[13] Konklusjon - inneholder avsluttende tanker, hva som fungerte og hva som kunne vært gjort annerledes.
  \item[] Bibliografi - inneholder alle kilder vi har benyttet i løpet av prosjektets periode.
  \item[] Vedlegg - inneholder kode som vi ønsket å forklare nærmere.
\end{itemize}
