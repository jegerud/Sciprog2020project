\documentclass[british]{ntnureport}

\title{An NTNU Report \LaTeX{} Document Class}
\shorttitle{An NTNU Report Document Class}
\author{Anders Kjelsrud \\
        Casper F. Gulbrandsen \\
		Kristian Jegerud \\ 
		Norges Teknisk-Naturvitenskapelige Universitet }
\shortauthor{CoPCSE$@$NTNU}
\date{CC-BY \ntnureportdate}

\addbibresource{report.bib}

\begin{document}

\begin{abstract}
    Emnet IMT3881 Vitenskapelig Programmering kombinerer realfaglige teknikker og høynivå programmering, og 
    gir studentene et innblikk i hvordan dette kan benyttes til å løse hverdagslige problemsstillinger. Et godt 
    eksempel på dette er bildebehandling, hvor numeriske løsninger av differensiallikninger og ikkelineære algebraiske
    likninger sammen med biblioteker fra Python kan benyttes til å behandle bilder. Hovedoppgaven vår har vært å
    implementere ulike anvendelser for dette, både for fargebilder og gråtonebilder. Vi har også skrevet et lite
    program med grafisk brukergrensesnitt (GUI) som viser eksempler på anvendelsene vi har implemntert. Ved å lese
    denne rapporten får du innsikt i hvordan vi har jobbet, hvilke teknikker som er brukt og hvordan disse er
    benyttet. All kildekoden ligger tilgjengelig på GitLab\cite{gitsource}. 
\end{abstract}

\section{Introduksjon}

The Community of Practice in Computer Science Education at NTNU (CoPCSE\-$@$NTNU)\footnote{\url{https://www.ntnu.no/wiki/display/copcse/Community+of+Practice+in+Computer+Science+Education+Home}} is hereby offering a template that should in principle be applicable for reports at all study levels. It is closely based on the standard \LaTeX{} \texttt{report} document class as well as previous thesis templates.

The purpose of the present document is threefold. It should serve (i) as a description of the document class, (ii) as an example of how to use it, and (iii) as a thesis template.
\input{chapters/2-task.tex}
\section{Report Structure}

The structure of the report, i.e., which chapters and other document elements that should be included, depends on the type of project it describes (development, research, investigation, consulting), and the diversity (narrow, broad). Thus, there are no exact rules for how to do it, so whatever follows should be taken as guidelines only.

A report, like a book, can typically be divided into three parts: front matter, body matter, and back matter. Of these, the body matter is by far the most important one, and also the one that varies the most between report types.

\subsection{Front Matter}
\label{sec:frontmatter}

The front matter is everything that comes before the main part of the report. It is common to use roman page numbers for this part to indicate this. The minimum required front matter consists of a title page and an abstract:

\begin{description}
    \item[Title:] The title should, at minimum, include the report title, authors and a date.
    \item[Abstract:] The abstract should be an extremely condensed version of the report. Think one sentence with the main message from each of the chapters of the body matter as a starting point. \textcite{landes1951scrutiny} have given some very nice instructions on how to write a good abstract.
\end{description}

\subsection{Body Matter}

The body matter consists of the main chapters of the report. There is a great diversity in the structure chosen for different report types. Common to almost all is that the first chapter is an introduction, and that the last one is a conclusion followed by the bibliography.

\subsubsection{Development Project}
\label{sec:development}

For development projects, the main task is to develop something, typically a software prototype, for an `employer' (e.g., an external company or a research group). A report describing such a project is typically structured as a software development report with more or less the following chapters:

\begin{description}
    \item[Introduction:] The introduction of the report should take the reader all the way from the big picture and context of the project to the concrete task that has been solved in the report. A nice skeleton for a good introduction was given by \textcite{claerbout1991scrutiny}: \emph{review–claim–agenda}. In the review part, the background of the project is covered. This leads up to your claim, which is typically that some entity (software, device) or knowledge (research questions) is missing and sorely needed. The agenda part briefly summarises how your report contributes.
    \item[Requirements:] The requirements chapter should lead up to a concrete description of both the functional and non-functional requirements for whatever is to be developed at both a high level (use cases) and lower levels (low level use cases, requirements). If a classical waterfall development process is followed, this chapter is the product of the requirement phase. If a more agile model like, e.g., SCRUM is followed, the requirements will appear through the project as, e.g., the user stories developed in the sprint planning meetings.
    \item[Technical design:] The technical design chapter describes the big picture of the chosen solution. For a software development project, this would typically contain the system arcitechture (client-server, cloud, databases, networking, services etc.); both how it was solved, and, more importantly, why this architecture was chosen.
    \item[Development Process:] In this chapter, you should describe the process that was followed. It should cover the process model, why it was chosen, and how it was implemented, including tools for project management, documentation etc. Depending on how you write the other chapters, there may be good reasons to place this chapters somewhere else in the report.
    \item[Implementation:] Here you should describe the more technical details of the solution. Which tools were used (programming languages, libraries, IDEs, APIs, frameworks, etc.). It is a good idea to give some code examples. If class diagrams, database models etc. were not presented in the technical design chapter, they can be included here.
    \item[Deployment:] This chapter should describe how your solution can be deployed on the employer's system. It should include technical details on how to set it up, as well as discussions on choices made concerning scalability, maintenance, etc.
    \item[Testing and user feedback:] This chapter should describe how the system was tested during and after development. This would cover everything from unit testing to user testing; black-box vs. white-box; how it was done, what was learned from the testing, and what impact it had on the product and process.
    \item[Discussion:] Here you should discuss all aspect of your report and project. How did the process work? Which choices did you make, and what did you learn from it? What were the pros and cons? What would you have done differently if you were to undertake the same project over again, both in terms of process and product? What are the societal consequences of your work?
    \item[Conclusion:] The conclusion chapter is usually quite short – a paragraph or two – mainly summarising what was achieved in the project. It should answer the \emph{claim} part of the introduction. It should also say something about what comes next (`future work').
    \item[Bibliography:] The bibliography should be a list of quality-assured peer-reviewed published material that you have used throughout the work with your report. All items in the bibliography should be referenced in the text. The references should be correctly formatted depending on their type (book, journal article, conference publication, report etc.). If \texttt{biblatex} is correctly used as proposed by this template, the formatting will be taken care of automatically. The bibliography should not contain links to arbitrary dynamic web pages where the content is subject to change at any point of time. Such links, if necessary, should rather be included as footnotes throughout the document. The main point of the bibliography is to back up your claims with quality-assured material that future readers will actually be able to retrieve years ahead.
\end{description}

\subsubsection{Research Project}
\label{sec:resesarch}

For research projects, the main task is to gain knew knowledge about something. A report describing such a project is typically structed as an extended form of a scientific paper, following the so-called IMRaD (Introduction, Method, Results, and Discussion) model:

\begin{description}
    \item[Introduction:] See \cref{sec:development}.
    \item[Background:] Research projects should always be based on previous research on the same and/or related topics. This should be described as a background to the report with adequate bibliographical references. If the material needed is too voluminous to fit nicely in the review part of the introduction, it can be presented in a separate background chapter. This chapter is also often called `Theory'.
    \item[Method:] The method chapter should describe in detail which activities you undertake to answer the research questions presented in the introduction, and why they were chosen. This includes detailed descriptions of experiments, surveys, computations, data analysis, statistical tests etc.
    \item[Results:] The results chapter should simply present the results of applying the methods presented in the method chapter without further ado. This chapter will typically contain many graphs, tables, etc. Sometimes it is natural to discuss the results as they are presented, combining them into a `Results and Discussion' chapter, but they are more often kept separate.
    \item[Discussion:] See \cref{sec:development}.
    \item[Conclusion:] See \cref{sec:development}.
    \item[Bibliography:] See \cref{sec:development}.
\end{description}

\subsection{Back Matter}

Material that does not fit elsewhere, but that you would still like to share with the readers, can be put in appendices. See \cref{app:additional}.

\section{Conclusion}

You definitely should use the \texttt{ntnureport} \LaTeX{} document class for your report.


\printbibliography{report}

\appendix

\section{Additional Material}
\label{app:additional}

Additional material that does not fit in the main thesis but may still be relevant to share, e.g., raw data from experiments and surveys, code listings, additional plots, pre-project reports, project agreements, contracts, logs etc., can be put in appendices. Simply issue the command \texttt{\textbackslash appendix} in the main \texttt{.tex} file, and make one section per appendix.

If the appendix is in the form of a ready-made PDF file, it should be supported by a small descriptive text, and included using the \texttt{\textbackslash includepdf[]\{\}} command provided by the \texttt{pdfpages} package. Use the option \texttt{[pages=-]} to include all pages of the PDF document, or, e.g., \texttt{[pages=2-4]} to include only the given page range.

\end{document}
