\begin{titlepage}
\vbox{ }
\vbox{ }
\begin{center}

\includegraphics[width=0.50\textwidth]{bilder/NTNU_logo.png}\\[0.7cm]
\textsc{\LARGE IMT3881 - Vitenskapelig Programmering}\\[0.4cm]
\vbox{ }

{\huge \bfseries Poisson Image Editing - Prosjektoppgave}\\[0.4cm]
\HRule \\[1.5cm]

\large
Anders Kjelsrud 509238\\
Casper F. Gulbrandsen 509242 \\
Kristian Jegerud 509249\\[1.8cm]

\begin{abstract}
Emnet IMT3881 Vitenskapelig Programmering kombinerer realfaglige teknikker med høynivå programmering, og gir studentene et innblikk i hvordan dette kan benyttes til å løse reelle praktiske problemsstillinger \cite{PCP}. Et godt eksempel på dette er bildebehandling, hvor numeriske løsninger av differensiallikninger og ikkelineære algebraiske likninger sammen med biblioteker fra Python kan benyttes til å behandle bilder. Oppgaven vår har vært å implementere ulike anvendelser for dette, både for fargebilder og gråtonebilder. Vi har også laget et grafisk brukergrensesnitt (GUI) som viser eksempler på anvendelsene vi har implementert. Ved å lese denne rapporten får du innsikt i hvordan vi har jobbet, hvilke teknikker som er brukt og hvordan disse er benyttet. All kildekoden ligger tilgjengelig på GitLab\cite{src}. 
\end{abstract}
\vfill

{\large Våren 2020}
\end{center}
\end{titlepage}